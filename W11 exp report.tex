\documentclass[12pt,a4paper]{article}
\usepackage{kotex}
\usepackage{graphicx}
\usepackage{hyperref}
\usepackage{indentfirst}
\usepackage{subcaption}
\usepackage{multirow}
\usepackage{flafter}
\usepackage{tikz}
\usetikzlibrary{arrows.meta, intersections, decorations.markings,
    positioning, backgrounds, through, calc, angles, quotes}
\setlength{\parskip}{2mm}
\usepackage{amsmath}
\usepackage[top=3cm, bottom=2.54cm, left=2.54cm, right=2.54cm]{geometry}
\usepackage[yyyymmdd]{datetime}
\renewcommand{\dateseparator}{-}
\usepackage{array}
\newcolumntype{L}[1]{>{\raggedright\let\newline\\\arraybackslash\hspace{0pt}}m{#1}}
\newcolumntype{C}[1]{>{\centering\let\newline\\\arraybackslash\hspace{0pt}}m{#1}}
\newcolumntype{R}[1]{>{\raggedleft\let\newline\\\arraybackslash\hspace{0pt}}m{#1}}

\begin{document}
\begin{titlepage}
    \centering
    \begin{tabular}{|C{15cm}|}
        \hline
        \rule{0in}{6ex}
        {\huge 물리학 및 실험 1\par} \\ 
        {\large 힘센서를 활용한 충격량과 작용 반작용 법칙\par} \\
        \hline
    \end{tabular} \\
    \vspace{5cm}
    \includegraphics[height=7.36cm]{logo.png}\par
    \vspace{3cm}
    \begin{tabular}{|l|l|l|l|l|l|}
        \hline
        과목 & \multicolumn{5}{l|}{물리학및실험1} \\
        \hline
        담당교수 & \multicolumn{2}{l|}{전계진} & 담당조교 & \multicolumn{2}{l|}{} \\
        \hline
        조 및 조원 & \multicolumn{5}{l|}{2조, 김민수 김민규 김민서 김백준 김연주} \\
        \hline
        제출일 & \multicolumn{5}{l|}{\today} \\
        \hline
        작성자 & 김민수 & 학번 & 20518009 & 학과 & 정보보호 \\
        \hline
    \end{tabular}
\end{titlepage}
\section{실험목적}
\begin{itemize}
    \item 스마트 카트를 이용하여 충돌실험을 하고 충돌 전후에 카트의 운동량 변화가
        카트가 받는 충격량과 같은지 확인한다.
    \item 충돌하는 두 물체 사이에 작용하는 힘이 작용반작용의 법칙을 따르는지
        확인한다.
\end{itemize}
\section{서론}
\begin{itemize}
    \item 일상에서 물체가 충돌하는 일은 자주 발생한다. 두 물체가 서로 충돌할 때 매우
        짧은 시간 동안 순간적으로 큰 힘이 작용한다. 이 때문에 뉴턴의 운동 제2 법칙으
        로 이러한 현상을 설명하기는 쉽지 않다.
    \item 물리학에서는 이와 같은 충돌 현상을 설명하기 위해 운동량과 충격량 개념을 도
        입한다. 충돌이나 폭발과 같은 현상에서도 총운동량이 보존되고 각각의 물체가
        받은 충격량은 자신의 운동량 변화와 같다는 충격량-운동량 정리가 성립한다고
        알려져 있다.
    \item 우리는 힘 센서와 스마트 카트를 이용하여 물체가 서로 충돌하는 실험을 하고,
        충돌하는 두 물체 사이에 작용반작용의 법칙과 충격량-운동량 정리가 성립하는가
        를 알아볼 것이다.
    \item 힘 센서를 이용하여 충돌하는 두 물체 사이에 작용하는 힘을 측정하여 뉴턴의
        운동 제3 법칙인 작용반작용의 법칙이 성립하는가를 확인하고, 스마트 카트에
        내장된 속도 센서로 속도를 측정하여 충격량-운동량 정리가 성립하는가를 알아볼
        것이다.
\end{itemize}
\section{실험원리}
\subsection{운동량 및 운동량과 힘과의 관계}
운동량은 벡터 $\vec{p}$로 나타내며 질량과 속도의 곱으로 정의 한다.
$$\vec{p}=m\vec{v}$$
뉴턴의 제2법칙 “운동량의 시간변화율은 알짜 힘과 같다”
$$\Sigma\vec{F}=\frac{\Delta\vec{v}}{\Delta t} =
    \frac{\Delta m}{\Delta t}\vec{v}+m\frac{\Delta\vec{v}}{\Delta t} = 
    \frac{\Delta m}{\Delta t}\vec{v}+m\vec{a}$$
질량이 일정한 경우 $\Sigma\vec{F}=m\vec{a}$
\subsection{운동량 충격량 정리}
충돌하는 동안에는 물체에 큰 힘이 작용하여 물체가 변형이 된다.
$$\Sigma\vec{F}=\frac{d\vec{p}}{dt} \Rightarrow
    d\vec{p} = \Sigma\vec{F}dt \Rightarrow
    \Delta\vec{p} = \vec{p}_2 - \vec{p}_1 = \int^{t_2}_{t_1}\Sigma\vec{F}dt$$
충격량(Impulse): 물체가 받는 충격의 정도를 나타냄
\begin{figure}[h!]
    \begin{subfigure}{0.3\textwidth}
        $\vec{I}\equiv $
    \end{subfigure}
\end{figure}
\section{실험장치 및 방법}
\section{실험 결과 및 분석}
\section{실험 고찰}
\end{document}