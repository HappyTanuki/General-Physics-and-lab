\documentclass[12pt,a4paper]{article}
\usepackage{kotex}
\usepackage{graphicx}
\usepackage{hyperref}
\usepackage{indentfirst}
\usepackage{subcaption}
\usepackage{multirow}
\usepackage{flafter}
\usepackage{tikz}
\usepackage{wrapfig}
\usepackage{spreadtab}
\usetikzlibrary{arrows.meta, intersections, decorations.markings,
    positioning, backgrounds, through, calc, angles, quotes}
\setlength{\parskip}{2mm}
\usepackage{amsmath}
\usepackage[top=3cm, bottom=2.54cm, left=2.54cm, right=2.54cm]{geometry}
\usepackage[yyyymmdd]{datetime}
\renewcommand{\dateseparator}{-}
\usepackage{array}
\newcolumntype{L}[1]{>{\raggedright\let\newline\\\arraybackslash\hspace{0pt}}m{#1}}
\newcolumntype{C}[1]{>{\centering\let\newline\\\arraybackslash\hspace{0pt}}m{#1}}
\newcolumntype{R}[1]{>{\raggedleft\let\newline\\\arraybackslash\hspace{0pt}}m{#1}}

\begin{document}
\begin{titlepage}
    \centering
    \begin{tabular}{|C{15cm}|}
        \hline
        \rule{0in}{6ex}
        {\huge 물리학 및 실험 1\par} \\ 
        {\large 로터리 모션 센서를 활용한 관성모멘트 측정\par} \\
        \hline
    \end{tabular} \\
    \vspace{5cm}
    \includegraphics[height=7.36cm]{logo.png}\par
    \vspace{3cm}
    \begin{tabular}{|l|l|l|l|l|l|}
        \hline
        과목 & \multicolumn{5}{l|}{물리학및실험1} \\
        \hline
        담당교수 & \multicolumn{2}{l|}{전계진} & 담당조교 & \multicolumn{2}{l|}{} \\
        \hline
        조 및 조원 & \multicolumn{5}{l|}{2조, 김민수 김민규 김민서 김백준 김연주} \\
        \hline
        제출일 & \multicolumn{5}{l|}{\today} \\
        \hline
        작성자 & 김민수 & 학번 & 20518009 & 학과 & 정보보호 \\
        \hline
    \end{tabular}
\end{titlepage}
\section{서론}
\begin{itemize}
    \item 크기와 모양을 갖는 실제 물체는 힘을 가하면 회전운동을 한다.
    \item 직선운동 하는 물체의 질량이 관성의 역할을 하듯이 회전운동하는 물체의
        관성의 역할을 하는 것이 관성 모멘트이다.
    \item 강체의 회전 관성은 물체의 질량 뿐 아니라 질량이 회전축에 대해 어떻게
        분포되어 있느냐에 따라 달라진다.
    \item 여러 가지 형태의 강체들의 관성모멘트를 측정하여 관성모멘트 개념과 정의를
        이해한다.
\end{itemize}
\section{실험목적}
강체는 기하학적인 형태에 따라 관성모멘트가 다르다. 여러가지 형태의 강체들의
관성모멘트를 측정하여 관성모멘트의 개념과 정의를 이해한다
\section{실험원리}
\section{실험기구 및 장치}
\section{실험방법}
\section{실험 결과}
\section{고찰}
\subsection{오차원인}
\subsection{실험을 통해 배우게 된 것}
\subsection{실험원리의 실생활에서의 예}
\end{document}